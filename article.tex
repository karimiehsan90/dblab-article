\documentclass[12pt,onecolumn,a4paper]{article}
\usepackage{epsfig,graphicx,subfigure,amsthm,amsmath}
\usepackage{color,xcolor}     
\usepackage{xepersian}
\settextfont[Scale=1.2]{BZAR.TTF}
\setlatintextfont[Scale=1]{times-new-roman.ttf}





\begin{document}
\title{عنوان مقاله} 
\author{میلاد رفیعی، کارشناسی دانشگاه علم وصنعت ایران واحد بهشهر\\
دانشجوی کارشناسی ارشد دانشگاه تفرش\\
مهندسی صنایع}
\date{\today}
\maketitle

\section{مقدمه} 
در این قسمت مطالب مربوط به بخش مقدمه آورده می شود.\\
این یک فرمت ساده است، شما می توانید با استفاده از دستورات لاتک و پاک کردن قسمت های نوشته شده، فایل مدنظر خود را بسازید.\\
متن زیر از سایت ویکی پدیا کپی پیست شده است و ممکن است بعضی جاها خوب نمایش داده نشود


سایت \lr{overleaf.com} یک ویرایشگر آنلاین \lr{\LaTeX{}} است که این اجازه را به ما می دهد در هرجایی تنها با داشتن اینترنت به پروژه های خود دسترسی داشته باشیم.\\
با نگارش متن علمی در فضای ابری (آنلاین)، علاوه بر داشتن امکان کار مشترک بر روی مقاله، ما از اجبار به استفاده از یک کامپیوتر مشخص نیز رها خواهیم شد. به عبارت دیگر هر کامپیوتر متصل به اینترنت کامپیوتر کار علمی شما خواهد بود. این موضوع به طور ویژه برای دانشجویانی که بر روی کامپیوتر های عمومی دانشگاه و آزمایشگاه تحقیقاتی خود مسیر نگارش متن علمی را به پیش می برند بسیار مفید خواهد بود. زیرا این دسته خواهند توانست بخشی از کار را در یک کامپیوتر به پیش برده و چند دقیقه بعد در یک کامپیوتر دیگر ادامه آن را پیگیری نمایند. از این جهت، در کنار امکان همکاری چند مولف بر روی مقاله، مستقل شدن نگارنده از کامپیوتر مورد استفاده مزیت دیگر نگارش ابری مقاله خواهد بود.\\

\section{ مهندسی صنایع}
مهندسی صنایع   از شاخه های مهندسی است، که می کوشد با تلفیق دانش مهندسی، ریاضیات، اقتصاد و
مدیریت، کارایی سیستم های تولیدی، فرایندها و سازمان ها را بهبود دهد.
مهندسی صنایع بطور کلی اثربخشی، کارایی، تطبیق پذیری، پاسخ گویی، کیفیت و بهبود مستمر کالاها، خدمات و سودآوری را مدنظر قرار می دهد. با
تحلیل و مدیریت و برنامه ریزی دقیق به این موارد میپردازد.
مهندسی صنایع عنوانی برای بیان رویکردی در علم مدیریت سازمان ها و صنایع می باشد.
نام فارسی این رشته به عنوان معادل درست انگلیسی آن عبارت مهندسی صنعتی است که توسط فردریک تیلور به کار برده شد.

\section{تاریخچه مهندسی صنایع}
در زمان تیلور که تعدادی از موضوعات مدیریتی توسط وی ارائه گردید توسط افراد دیگر برای اینکه اهمیت این موضوعات و نیز جایگاه دست اندرکاران
مربوطه از رشته های مهندسی کمتر نمود داده نشود و مهندسین برای آنها نیز اهمیت قایل شوند از اصطلاح مهندسی صنایع استفاده شد. موضوعات با
نام مهندسی صنایع بخشی از موضوعات علم مدیریت می باشد. فردریک تیلور (١٩١٥–١٨٥٦ (بنیانگذار علم مدیریت را پدر مهندسی صنایع می دانند.
تعاریفی که از مهندسی صنایع ارائه می شود مانند اینکه رشته ای است که با طراحی، بهبود، و پیاده سازی سیستمای یکپارچه از افراد، مواد، اطلاعات،
تجهیزات و انرژی مرتبط می باشد، بیانی از علم مدیریت می باشد. مهندسی صنایع آن گونه که در ایران تدریس می شود مبتنی بر بهینه سازی،
بهینه سازی خطی (تحقیق در عملیات) کارسنجی و زمان سنجی می باشد.





\section{مهندسی صنایع در ایران}
دانشکده مهندسی صنایع دانشگاه صنعتی شریف به عنوان اولین دانشکده مهندسی صنایع کشور در سال ١٣٤٧ با پذیرش ٣٩ دانشجو آغاز به کار کرد.
در سال ١٣٥٢ نیز دانشکده مهندسی صنایع دانشگاه علم و صنعت ایران آغاز به کار کرد؛ و در سال ١٣٥٥ گروه مستقلی تحت عنوان مهندسی صنایع
در دانشگاه صنعتی امیرکبیر (پلی تکنیک تهران) تشکیل شد و اولین گروه فارغ التحصیلان ٔ دانشکده مهندسی صنایع این دانشگاه در سال ١٣٦٢ وارد
ٔ جامعه صنعتی کشور گردید. با گذشت زمان، این رشته به عنوان رشته ای در دانشگاه های مختلف کشور جایگاه خود را پیدا کرده است.










\section{فرمول نویسی}
\begin{align*}
\max z =&\sum_{k=1}^{k_1}(c_{1}X^{1,K})\lambda_{1k}+\sum_{k=1}^{k_1}(c_{2}X^{2,K})\lambda_{2k}+\cdots+\sum_{k=1}^{k_n}(c_{n}X^{n,K})\lambda_{nk} \\
st:\qquad&\sum_{k=1}^{k_1} (A_{1}X^{1,k})\lambda_{1k}+\sum_{k=1}^{k_2}(A_{2}X^{2,k})\lambda_{2k}+\cdots+\sum_{k=1}^{k_n}(A_{n}X^{n,k})\lambda_{nk}\leq b_{0} \\
&\sum_{k=1}^{k_1}\lambda_{1k}=1\\
&\sum_{k=1}^{k_1}\lambda_{1k}=1\\
&\qquad\vdots\\
&\sum_{k=1}^{k_1}\lambda_{1k}=1\\
&\lambda_{jk}\geq 0\\
\end{align*}



\section{نتایج}
در این قسمت نتایج نوشته می شود.\\
مهندسین صنایع در سالهای اخیر در طیف وسیعی از مشاغل به کار گرفته شده اند. علاوه بر زمینه های تخصصی، فارغ التحصیلان این رشته در
زمینه هایی مانند مهندسی نرم افزار، فناوری اطلاعات، هوش کسب و کار، داده کاوی، علم داده و کلان داده نیز عملکرد موفقی داشته اند. زمینه های
کاری دیگر این رشته مدیریت تولید و خط تولید، برنامه ریزی خطی و غیر خطی، آینده پژوهی و مدیریت فناوری و نوآوری که دو رشته بسیار خوب
در کشور های اروپایی می باشند.

\begin{thebibliography}{99}
\bibitem{}
سایت فرادرس
\bibitem{}
سایت ویکی پدیا فارسی




\end{thebibliography}









\end{document}


